\section*{Приложение А. Исходный код программы}
\addcontentsline{toc}{section}{\tocsecindent{Приложение А. Исходный код программы}}

\begin{verbatim}

package net.scienceontheweb.djbelyak.NeuronNetwork.model;

import java.util.Random;

/**
 * Класс, описывающий нейрон
 */
public class Neuron {
	/**
	 * Вектор входных сигналов нейрона
	 */
	private double[] x;
	/**
	 * Вектор весовых коэффициентов
	 */
	private double[] w;
	/**
	 * Параметр наклона сигмоидальной функции
	 */
	private double a = 0.5;
	/**
	 * Количество входных сигналов
	 */
	private int n = 1;

	/**
	 * Функция инициализации синаптических весов случайными элементами
	 * @param aParameter Максимальное значение синаптического веса
	 */
	public void initWeights(double aParameter) {
		Random rand = new Random();
		for (int i=0; i<this.n; i++)
			w[i]=rand.nextDouble()*aParameter;
	}

	/**
	 * Корректировка синаптических весов при обучении
	 * @param aEta Скорость обучения
	 * @param aDelta Корректируемая ошибка между выходным и требуемым сигналом
	 */
	public void correctWeights(double aEta, double aDelta) {
		for (int i=0; i<this.n; i++)
			w[i]+=aEta*aDelta*x[i];
	}

	/**
	 * Конструктор
	 * @param aN Количество синапсов нейрона
	 */
	public  Neuron(int aN) {
		n = aN;
		w = new double[n];
		x = new double[n];
	}

	/**
	 * Метод получения выходного значения нейрона
	 */
	public double getY() {
		double sum = 0.0;
		for (int i = 0; i<n; i++)
			sum +=w[i]*x[i];
		return 1/(1+Math.exp(-a*sum));
	}

	public void setX(double[] aX) {
		this.x = aX;
	}

	public void setA(double aA) {
		this.a = aA;
	}
}

package net.scienceontheweb.djbelyak.NeuronNetwork.model;

import net.scienceontheweb.djbelyak.NeuronNetwork.model.Neuron;

/**
 * Класс нейронной сети
 */
public class Network {
	/**
	 * Количество входных нейронов
	 */
	private int nIn;
	/**
	 * Количество выходных нейронов
	 */
	private int nOut;
	/**
	 * Параметр наклона сигмоидальной функции
	 */
	private double a;
	/**
	 * Скорость обучения нейронной сети
	 */
	private double eta;
	/**
	 * Коллекция нейронов слоя
	 */
	private Neuron neurons[];

	/**
	 * Конструктор
	 * @param aIn Размерность входного вектора
	 * @param aOut Размерность выходного вектора
	 */
	public Network(int aIn, int aOut) {
		nIn = aIn;
		nOut = aOut;
		
		neurons = new Neuron[nOut];
		for (int i = 0; i < nOut; i++)
		{
			neurons [i] = new Neuron(nIn);
			neurons [i].initWeights(0.15);
		}
	}

	/**
	 * Метод обработки входного образа
	 * @param aXIn Входной вектор
	 */
	public double[] work(double[] aXIn) {
		double y[] = new double [nOut];
		for (int i = 0; i < nOut; i++)
		{
			neurons[i].setX(aXIn);
			y[i] = neurons[i].getY();
		}
		return y;
	}

	/**
	 * Метод обучения нейронной сети
	 * @param aXIn Входной вектор
	 * @param aYOut Требуемый выходной вектор
	 */
	public double teach(double[] aXIn, double[] aYOut) {
		double d, sr_d=0.0;
        
        double[] t = this.work(aXIn);
        // подстройка весов каждого нейрона
        for (int j = 0; j < nOut; j++) 
        {
        	d = aYOut[j] - t[j];
            neurons[j].correctWeights(eta, d);
            sr_d+=d;
        }
        sr_d/=(double)nOut;
        return sr_d;
        
	}

	public int getNIn() {
		return this.nIn;
	}

	public int getNOut() {
		return this.nOut;
	}

	public void setA(double aA) {
		this.a = aA;
	}

	public double getA() {
		return this.a;
	}

	public void setEta(double aEta) {
		this.eta = aEta;
	}

	public double getEta() {
		return this.eta;
	}
}

package net.scienceontheweb.djbelyak.NeuronNetwork.model;

import java.awt.Container;
import java.awt.Image;
import java.awt.MediaTracker;
import java.awt.image.PixelGrabber;
import java.io.File;
import java.io.FilenameFilter;
import java.util.logging.Level;
import java.util.logging.Logger;

/**
 * Класс учителя нейронной сети
 */
public class Teacher {
	/**
	 * Средняя ошибка на каждом цикле обучения
	 */
	private double[] errors;
	Network network;

	/**
	 * Конструктор
	 * @param aNetwork Обучаемая сеть
	 */
	public Teacher(Network aNetwork) {
		network = aNetwork;
	}

	/**
	 * Метод генерации входного вектора для персептрона
	 * @param aIn Входное изображение
	 */
	public double[] getInVector(int[] aIn) {
		double[] x = new double[aIn.length];
        for (int i = 0; i < aIn.length; i++) {
            if (aIn[i] == -1) x[i] = 0.0; else x[i] = 1.0;
        }
        return x;
	}

	/**
	 * Метод генерации контрольного выходного вектора
	 * @param aIn Номер активного выходного нейрона
	 */
	public double[] getOutVector(int aIn) {
		double[] y = new double [network.getNOut()];
        for (int i = 0; i < y.length; i++) {
            if (i == aIn) y[i] = 1.0; else y[i] = 0.0;
        }
        return y;
	}

	/**
	 * Метод обучения
	 * @param aPath Путь к файлам с обучающими материалами
	 * @param aN Количество циклов обучения
	 */
	public void teach(String aPath, int aN) {
		class JPGFilter implements FilenameFilter {
            public boolean accept(File dir, String name) {
                return (name.endsWith(".jpg"));
            }
        }     
 
        // загрузка всех тестовых изображений в массив img[]
        String[] list = new File(aPath + "/").list(new JPGFilter());
        Image[] img = new Image[list.length];        
        MediaTracker mediaTracker = new MediaTracker(new Container());        
        int i = 0;
        for (String s: list) {
            img[i] = java.awt.Toolkit.getDefaultToolkit().createImage(aPath + "/" + s);
 
            mediaTracker.addImage(img[i], 0);
            try {
                mediaTracker.waitForAll();
            } catch (InterruptedException ex) {
                Logger.getLogger(Teacher.class.getName()).log(Level.SEVERE, null, ex);
            }
 
            i++;
        }
 
 
        // получение пиксельных массивов каждого изображения
        // и обучение n раз каждой выборке
        PixelGrabber pg;
        int[] pixels;
		double[] y;
		double[] x;
		errors = new double[aN];
        int w, h;
        for (i=0; i<aN; i++)
        {
            for (int j = 0; j < img.length; j++) {
                w = img[j].getWidth(null);
                h = img[j].getHeight(null);
 
                if (w*h > network.getNIn()) continue;
 
                pixels = new int[w*h];
                pg = new PixelGrabber(img[j], 0, 0, w, h, pixels, 0, w);
                try {
                    pg.grabPixels();
                } catch (InterruptedException ex) {
                    Logger.getLogger(Teacher.class.getName()).log(Level.SEVERE, null, ex);
                }
 
                // получение векторов и обучение перцептрона
                x = getInVector(pixels);
                y = getOutVector(Integer.parseInt(String.valueOf(list[j].charAt(0))));
                errors[i] = network.teach(x, y);
            }
        }
	}

	public double[] getErrors() {
		return this.errors;
	}
}

package net.scienceontheweb.djbelyak.NeuronNetwork.view;

import java.awt.Container;
import java.awt.Image;
import java.awt.MediaTracker;
import java.awt.event.ActionEvent;
import java.awt.event.ActionListener;
import java.awt.image.PixelGrabber;
import java.util.logging.Level;
import java.util.logging.Logger;

import javax.swing.JButton;
import javax.swing.JFileChooser;
import javax.swing.JFrame;
import javax.swing.JLabel;
import javax.swing.JSpinner;
import javax.swing.JTextField;

import org.jfree.chart.ChartFactory;
import org.jfree.chart.ChartFrame;
import org.jfree.chart.JFreeChart;
import org.jfree.chart.plot.PlotOrientation;
import org.jfree.data.xy.XYSeries;
import org.jfree.data.xy.XYSeriesCollection;

import net.scienceontheweb.djbelyak.NeuronNetwork.model.*;

@SuppressWarnings("serial")
public class NeuronNetworkFrame extends JFrame {


	private JLabel JLabelEta;
	private JLabel JLabelA;
	private JLabel JLabelN;
	private JTextField JTextFieldEta;
	private JTextField  JTextFieldA;
	private JSpinner JSpinnerN;
	private JButton JButtonStud;
	private JButton JButtonCheck;
	private JLabel JLabelRes;
	private Network network;
	private Teacher teacher;

	public NeuronNetworkFrame()
	{
		super();
		network = new Network(64*64,10);
		teacher = new Teacher(network);
		
	    this.setSize(501, 300);
	    this.getContentPane().setLayout(null);
	    this.setTitle("Алгоритмы обучения и топологии нейронных сетей");
	    this.add(getJLabelEta(),null);
	    this.add(getJTextFieldEta(),null);
	    this.add(getJLabelA(),null);
	    this.add(getJTextFieldA(),null);
	    this.add(getJLabelN(),null);
	    this.add(getJSpinnerN(),null);
	    this.add(getJButtonStud(),null);
	    this.add(getJButtonCheck(),null);
	    this.add(getJLabelRes(),null);
	}
	
	private javax.swing.JLabel getJLabelEta() {
	      if(JLabelEta == null) {
	    	  JLabelEta = new javax.swing.JLabel();
	         JLabelEta.setBounds(0, 0, 350, 20);
	    	  JLabelEta.setText("Скорость обучения ИНС (eta):");
	      }
	      return JLabelEta;
	   }
	
	
	private javax.swing.JLabel getJLabelA() {
	      if(JLabelA == null) {
	    	  JLabelA = new javax.swing.JLabel();
	    	  JLabelA.setBounds(0, 20, 350, 20);
	    	  JLabelA.setText("Параметр наклона сигмоидальной функции (a):");
	      }
	      return JLabelA;
	   }
	
	private javax.swing.JLabel getJLabelN() {
	      if(JLabelN == null) {
	    	  JLabelN = new javax.swing.JLabel();
	    	  JLabelN.setBounds(0, 40, 350, 20);
	    	  JLabelN.setText("Количество шагов обучения (n):");
	      }
	      return JLabelN;
	   }
	
	private javax.swing.JTextField getJTextFieldEta(){
		if(JTextFieldEta == null) {
			JTextFieldEta = new javax.swing.JTextField();
			JTextFieldEta.setBounds(350, 1, 148, 18);
			JTextFieldEta.setText("0.15");
			
		}
		return JTextFieldEta;
	}
	
	private javax.swing.JTextField getJTextFieldA(){
		if(JTextFieldA == null) {
			JTextFieldA = new javax.swing.JTextField();
			JTextFieldA.setBounds(350, 21, 148, 18);
			JTextFieldA.setText("0.5");
		}
		return JTextFieldA;
	}
	
	private javax.swing.JSpinner getJSpinnerN(){
		if(JSpinnerN == null) {
			JSpinnerN = new javax.swing.JSpinner();
			JSpinnerN.setBounds(350, 41, 148, 18);	
			JSpinnerN.setValue(100);
			
		}
		return JSpinnerN;
	}
	
	private javax.swing.JButton getJButtonStud(){
		if(JButtonStud == null) {
			JButtonStud = new javax.swing.JButton();
			JButtonStud.setBounds(0, 60, 250, 20);	
			JButtonStud.setText("Обучить нейронную сеть");
			JButtonStud.addActionListener(new  ActionListener() {
				@Override
	               public void actionPerformed(ActionEvent e) {
						int n = Integer.parseInt(JSpinnerN.getValue().toString());
						network.setA(Double.parseDouble(JTextFieldA.getText()));
						network.setEta(Double.parseDouble(JTextFieldEta.getText()));
	                    teacher.teach("data", n);
	                    
	                 // create a dataset...
	                 
	                    XYSeries series1 = new XYSeries("Ошибка");
	                    for (int i = 0; i < n; i++)
	                    	series1.add(i, teacher.getErrors()[i]);
	                    
	                    
	                    XYSeriesCollection dataset = new XYSeriesCollection();
	                    dataset.addSeries(series1);
	                    // create the chart...
	                    JFreeChart chart = ChartFactory.createXYLineChart("Зависимость ошибки от номера итерации обучения", 
	                    		"Номер итерации", "Корректируемая ошибка", dataset, PlotOrientation.VERTICAL, true, true, true);
	     

	                    // create and display a frame...
	                    ChartFrame frame = new ChartFrame("Алгоритмы обучения и топологии нейронных сетей", chart);
	                    frame.pack();
	                    frame.setVisible(true);

	               }

			

	          });

			
		}
		return JButtonStud;
	}
	
	private javax.swing.JButton getJButtonCheck(){
		if(JButtonCheck == null) {
			JButtonCheck = new javax.swing.JButton();
			JButtonCheck.setBounds(250, 60, 250, 20);	
			JButtonCheck.setText("Проверить работу");
			JButtonCheck.addActionListener(new ActionListener(){

			
				public void actionPerformed(ActionEvent arg0) {
					JFileChooser fileopen = new JFileChooser();
					int ret = fileopen.showDialog(null, "Открыть файл");				
					if (ret == JFileChooser.APPROVE_OPTION) {
						
						
						MediaTracker mediaTracker = new MediaTracker(new Container());        
						
							Image	img = java.awt.Toolkit.getDefaultToolkit().createImage(fileopen.getSelectedFile().getAbsolutePath());
									mediaTracker.addImage(img, 0);
				            try {
				                mediaTracker.waitForAll();
				            } catch (InterruptedException ex) {
				                Logger.getLogger(Teacher.class.getName()).log(Level.SEVERE, null, ex);
				            }
					int w = img.getWidth(null);
	                int h = img.getHeight(null);
	 
	                int [] pixels = new int[w*h];
	                PixelGrabber pg = new PixelGrabber(img, 0, 0, w, h, pixels, 0, w);
	                try {
	                    pg.grabPixels();
	                } catch (InterruptedException ex) {
	                    Logger.getLogger(Teacher.class.getName()).log(Level.SEVERE, null, ex);
	                }
	 
	                // получение векторов и обучение перцептрона
	                double [] x = teacher.getInVector(pixels);
	                double [] y = network.work(x);
	                JLabelRes.setText("<html>");
	                for (int k=0; k<y.length;k++)
	                	JLabelRes.setText(JLabelRes.getText()+k+":"+y[k]+"<br>");
					
					}
					JLabelRes.setText(JLabelRes.getText()+"</html>");
				}});
			
		}
		return JButtonCheck;
	}
	
	private javax.swing.JLabel getJLabelRes() {
	      if(JLabelRes == null) {
	    	  JLabelRes = new javax.swing.JLabel();
	    	  JLabelRes.setBounds(0, 80, 500, 170);
	    	  JLabelRes.setText("");
	      }
	      return JLabelRes;
	   }
	
	
	/**
	 * @param args
	 */
	public static void main(String[] args) 
	{
		NeuronNetworkFrame m = new NeuronNetworkFrame();
		m.setVisible(true);

	}

}


\end{verbatim}
