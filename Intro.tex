\begin{center}
\section*{Введение}
\addcontentsline{toc}{section}{\tocsecindent{Введение}}
\end{center}


Длительный период биологической эволюции придал мозгу человека много качеств, которые отсутствуют как в машинах с архитектурой фон Неймана, так и в современных параллельных компьютерах.
К ним относятся:
\begin{itemize}
\item[-] массовый параллелизм
\item[-] распределенное представление информации и вычисления
\item[-] способность к обучению
\item[-] способность к обобщению
\item[-] адаптивность
\item[-] свойство контекстуальной обработки информации
\item[-] толерантность к ошибкам
\item[-] низкое энергопотребление
\end{itemize}
Можно предположить, что приборы и системы, построенные на тех же принципах, что и биологические нейроны, будут наследовать эти свойства.

Существует множество подходов к реализации искусственной нейронной сети (ИНС).
При этом большинство реализации основаны на функциональной парадигмы программирования (язык программирования {\it Lisp}).
При этом существует очень мало реализаций ИНС в объектно-ориентированной парадигме программирования.

Поэтому целью данной работы является реализация искусственной нейронной сети на языке {\it Java} для распознавания образов арабских цифр, представленных монохромными изображениями.
Для достижения поставленной цели необходимо выполнить следующие шаги:
\begin{enumerate}
\item изучить теоретические основы моделирования нейронных сетей;
\item реализовать ИНС на языке {\it Java} для распознавания арабских цифр;
\item исследовать влияние параметров нейронной сети на процесс распознавания символов.
\end{enumerate}


\newpage