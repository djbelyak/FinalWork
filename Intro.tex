\begin{center}
\section*{Введение}
\addcontentsline{toc}{section}{\tocsecindent{Введение}}
\end{center}


Длительный период биологической эволюции придал мозгу человека много качеств, которые отсутствуют как в машинах с архитектурой фон Неймана, так и в современных параллельных компьютерах.
К ним относятся:
\begin{itemize}
\item[-] массовый параллелизм
\item[-] распределенное представление информации и вычисления
\item[-] способность к обучению
\item[-] способность к обобщению
\item[-] адаптивность
\item[-] свойство контекстуальной обработки информации
\item[-] толерантность к ошибкам
\item[-] низкое энергопотребление
\end{itemize}
Можно предположить, что приборы и системы, построенные на тех же принципах, что и биологические нейроны, будут наследовать эти свойства.
В контексте данной работы детально будет рассмотрена программная реализация искусственной нейронной сети (ИНС).
В первой (теоретической) главе будут рассмотрены математические модели ИНС, топологии ИНС, модели и парадигмы обучения.
Во второй (практической) главе будет разработана программная модель ИНС с использованием объетно-ориентированного подхода для решения задачи распознавания образов.


\newpage