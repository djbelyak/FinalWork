\chapter{Практическая реализация искусственной нейронной сети}	
\section{Постановка задачи}

В качестве практической иллюстрации положений из теоретической части необходимо программно реализовать ИНС.
Для реализации искусственной нейронной сети необходимо воспользоваться объектно-ориентированным подходом и реализовать программу на языке {\it Java}.
Нейронная сеть на вход должна получать монохромное изображение арабской цифры размером 64 на 64 пиксела в формате {\it JPG}.
На выходе нейронной сети должна быть вероятность того, что на изображении присуствует та или иная цифра.

С точки зрения топологии нейронных сетей реализуемая нейронная сеть представляет собой однослойную сеть прямого распространения с 4096 входными нейронами и 10 выходными нейронами.
В качестве алгоритма обучения необходимо воспользоваться алгоритмом обратного распространения ошибки.

Должен быть реализован режим проверки работоспособности и режим обучения.
В режиме проверки работоспособности на вход нейронной сети подается файл с изображением, а на выходе должна быть рассчитана вероятность того, что в файле изображена какая-либо цифра
В режиме обучения сеть проходит заданое количество циклов обучения с определенными параметрами скорости обучения $\eta$ и параметром наклона сигмоидальной функции $a$.
По заврешении обучения строится график завимости корректируемой ошибки от количества итераций обучения.

