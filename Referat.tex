\begin{center}
\section*{Реферат}
\end{center}

\vspace{2em}
41 стр., 15 рис., 3 ист.

\vspace{2em}
РАССПОЗНАВАНИЕ ИЗОБРАЖЕНИЙ, ИСКУССТВЕННАЯ НЕЙРОННАЯ СЕТЬ, ТОПОЛОГИИ, АЛГОРИТМЫ ОБУЧЕНИЯ, ПАРАДИГМЫ ОБУЧЕНИЯ, ОБЪЕКТНО-ОРИЕНТИРОВАННОЕ ПРОГРАММИРОВАНИЕ, UML, JAVA

\vspace{2em}
Объектом исследования являются алгоритмы обучения и топологии искусственных нейронных сетей.

Цель работы --- программная реализация искусственной нейронной сети в объектно-ориентированной парадигме программирования.

В данной работе исследуются математические модели нейронов, топологии нейронных сетей, алгоритмы и парадигмы обучения нейронных сетей.
Проводится работа по созданию однослойной нейронной сети прямого распространения для распознавания образов арабских цифр.
Искусственная нейронная сеть создается в объектно-ориентированной парадигме программирования и на языке программирования Java.

Полученная искусственная нейронная сеть используется для исследования влияния количества циклов обучения, параметра скорости обучения и коэффициента наклона сигмоидальной функции на качество обучения и качество последующей работы нейронной сети.