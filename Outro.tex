\section*{\centering Заключение}
\addcontentsline{toc}{section}{\tocsecindent{Заключение}}

В ходе работы были получены следующие основные результаты и выводы.
\begin{enumerate}
\item Детально рассмотрены основы создания программных моделей нейронных сетей.

\item Рассмотрены общие принципы моделирования, топологии, методы и парадигмы обучения.

\item В ходе практической реализации создана объектно-ориентированная модель нейронной сети на языке {\it Java}.

\item Реализован алгоритм обратного распространения ошибки для решения задачи распознавания образов арабских цифр.

\item Выяснено, что однослойная сеть прямого распространения способна должным образом определять образы цифр из монохромных изображений малого разрешения (64 на 64 пикселя).

\item Опытным путем показано влияние параметра скорости обучения на необходимое для правильной работы число циклов обучения.

\item Опытным путем выяснено, что сигнал ошибки нейронной сети немонотонно убывает с ростом числа циклов обучения.

\item Показано, что коэффициент сигмоидальной функции влияет на ее наклон, а, следовательно, на толерантность работы сети.

\item На практике подтверждена способность к обучению --- нейронная сеть после обучения способна определять образ цифры.

\item На практике подтверждена толерантность к ошибкам --- регулируется коэффициентом наклона сигмоидальной функции.
\end{enumerate}

\newpage
