\begin{center}
\section*{Заключение}
\addcontentsline{toc}{section}{\tocsecindent{Заключение}}
\end{center}

В данной выпускной работе были детально рассмотрены основы создания программных моделей нейронных сетей.
Были рассмотрены общие принципы моделирования, топологии, методы и парадигмы обучения.

В ходе практической реализации была создана объектно-ориентированная модель нейронной сети на языке {\it Java}.
Был реализован алгоритм обратного распространения ошибки для решения задачи распознавания образов арабских цифр.
Было выяснено, что однослойная сеть прямого распространения способна должным образом определять образы цифр из монохромных изображений малого разрешения (64 на 64 пикселя).

Опятным путем было показано влияние параметра скорости обучения на необходимое для правильной работы число циклов обучения.
Также опытным путем было выяснено, что сигнал ошибки нейронной сети немонотонно убывает с ростом числа циклов обучения.
Было показано, что коэффициент сигмоидальной функции влияет на ее наклон, а, следовательно, на толерантность работы сети.

Таким образом, на практике были подтверждены следующие свойства искуственных нейронных сетей:
\begin{itemize}
\item[-] Способность к обучению --- нейронная сеть после обучения способна определять образ цифры;
\item[-] Толерантность к ошибкам --- регулируется коэффициентом наклона сигмоидальной функции.
\end{itemize}


\newpage